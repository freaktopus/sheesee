\chapter{System Analysis}

\section{Introduction}

System analysis is a critical phase in software development that involves understanding the requirements, constraints, and design considerations before implementation. This chapter presents a comprehensive analysis of the Sheesee image processing system, including its architecture, functional and non-functional requirements, and the hardware and software specifications needed for successful deployment.

\section{General Block Diagram}

The system architecture follows a pipeline-based design where data flows through distinct processing stages. Figure~\ref{fig:block_diagram} illustrates the high-level structure of the system.

\begin{figure}[h]
    \centering
    \includegraphics[width=0.9\textwidth]{block_diagram.png}
    \caption{General Block Diagram of the Sheesee System}
    \label{fig:block_diagram}
\end{figure}

The system consists of the following major components:

\begin{enumerate}
    \item \textbf{Input Module:} Accepts image files from the local filesystem
    \item \textbf{Image Processing Module:} Handles resizing and convolution operations
    \item \textbf{Authentication Module:} Manages OAuth 2.0 flow for Google API access
    \item \textbf{Data Transformation Module:} Converts pixel data to Google Sheets format
    \item \textbf{API Integration Module:} Handles communication with Google Sheets API
    \item \textbf{Output Module:} Provides user feedback and spreadsheet URL
\end{enumerate}

\section{Functional Requirements}

Functional requirements specify what the system must do. The following are the key functional requirements of the Sheesee system:

\subsection{FR1: Image Input and Validation}
\begin{itemize}
    \item The system shall accept image files in common formats (JPEG, PNG, BMP, GIF)
    \item The system shall validate the input file to ensure it is a readable image
    \item The system shall handle file read errors gracefully with appropriate error messages
\end{itemize}

\subsection{FR2: Image Resizing}
\begin{itemize}
    \item The system shall resize any input image to exactly 400×300 pixels
    \item The system shall use Lanczos3 interpolation algorithm for high-quality resizing
    \item The system shall maintain aspect ratio considerations during resizing
    \item The system shall support both upscaling and downscaling operations
\end{itemize}

\subsection{FR3: Convolution Processing}
\begin{itemize}
    \item The system shall apply a 3×3 sharpening kernel to the resized image
    \item The system shall perform convolution operations on all three color channels (RGB)
    \item The system shall clamp pixel values to the valid range [0, 255]
    \item The system shall handle boundary pixels appropriately (excluding 1-pixel border)
\end{itemize}

\subsection{FR4: User Authentication}
\begin{itemize}
    \item The system shall implement OAuth 2.0 authentication flow for Google API access
    \item The system shall provide a user consent URL for authorization
    \item The system shall handle callback from Google's authorization server
    \item The system shall store and validate access tokens
    \item The system shall support refresh token mechanism for extended access
    \item The system shall allow users to verify existing credentials before re-authentication
\end{itemize}

\subsection{FR5: Google Sheets Integration}
\begin{itemize}
    \item The system shall create or update Google Sheets with processed pixel data
    \item The system shall format each cell with appropriate RGB background color
    \item The system shall use batch update operations for efficient data upload
    \item The system shall set appropriate column widths and row heights for square pixels
    \item The system shall handle API rate limits and errors gracefully
\end{itemize}

\subsection{FR6: Command-Line Interface}
\begin{itemize}
    \item The system shall provide clear prompts for user interaction
    \item The system shall display progress information during processing
    \item The system shall output the final Google Sheets URL upon completion
    \item The system shall handle user input validation (yes/no responses)
\end{itemize}

\section{Non-Functional Requirements}

Non-functional requirements specify quality attributes and constraints of the system.

\subsection{NFR1: Performance}
\begin{itemize}
    \item Image resizing shall complete within 2 seconds for typical input images
    \item Convolution processing shall complete within 5 seconds for 400×300 images
    \item Google Sheets upload shall complete within 30 seconds for 120,000 cells
    \item The system shall efficiently utilize CPU and memory resources
\end{itemize}

\subsection{NFR2: Reliability}
\begin{itemize}
    \item The system shall handle network failures during API communication
    \item The system shall validate all user inputs before processing
    \item The system shall provide meaningful error messages for all failure scenarios
    \item The system shall not crash due to invalid input or network errors
\end{itemize}

\subsection{NFR3: Security}
\begin{itemize}
    \item The system shall securely store API credentials in environment variables
    \item The system shall use HTTPS for all API communications
    \item The system shall implement OAuth 2.0 standard for authentication
    \item The system shall not log or display sensitive tokens in plain text
\end{itemize}

\subsection{NFR4: Usability}
\begin{itemize}
    \item The CLI interface shall be intuitive and self-explanatory
    \item Error messages shall be clear and actionable
    \item The authentication flow shall be straightforward for non-technical users
    \item Documentation shall be comprehensive and easy to follow
\end{itemize}

\subsection{NFR5: Maintainability}
\begin{itemize}
    \item The codebase shall follow Rust best practices and idioms
    \item Modules shall be loosely coupled and highly cohesive
    \item Functions shall be well-documented with clear purposes
    \item The system shall be easily extensible for additional features
\end{itemize}

\subsection{NFR6: Portability}
\begin{itemize}
    \item The system shall run on Windows, Linux, and macOS platforms
    \item The system shall not depend on platform-specific features
    \item All dependencies shall be cross-platform compatible
\end{itemize}

\section{Software Requirements}

The following software components are required for development, deployment, and execution of the Sheesee system:

\subsection{Programming Language and Compiler}
\begin{itemize}
    \item \textbf{Rust:} Version 1.70 or higher with 2024 edition support
    \item \textbf{Cargo:} Rust's package manager and build system (bundled with Rust)
\end{itemize}

\subsection{Core Dependencies}

The project relies on several Rust crates (libraries) specified in \texttt{Cargo.toml}:

\begin{table}[h]
\centering
\caption{Software Dependencies and Their Purposes}
\label{tab:dependencies}
\begin{tabular}{|l|l|p{6cm}|}
\hline
\textbf{Crate} & \textbf{Version} & \textbf{Purpose} \\
\hline
axum & 0.8.7 & Web framework for OAuth callback server \\
\hline
tokio & 1.48.0 & Asynchronous runtime for async operations \\
\hline
image & 0.25.9 & Image loading, manipulation, and resizing \\
\hline
sheets & 0.7.0 & Google Sheets API client library \\
\hline
serde & 1.0.228 & Serialization/deserialization framework \\
\hline
serde\_json & 1.0.145 & JSON parsing and generation \\
\hline
dotenv & 0.15.0 & Environment variable management \\
\hline
\end{tabular}
\end{table}

\subsection{Development Tools}
\begin{itemize}
    \item \textbf{Text Editor/IDE:} Visual Studio Code, IntelliJ IDEA with Rust plugin, or any editor with Rust support
    \item \textbf{Git:} Version control system for source code management
    \item \textbf{rustfmt:} Code formatter for consistent Rust code style
    \item \textbf{clippy:} Linting tool for catching common mistakes
\end{itemize}

\subsection{Runtime Environment}
\begin{itemize}
    \item \textbf{Operating System:} Windows 10/11, Linux (Ubuntu 20.04+), or macOS 11+
    \item \textbf{Network Access:} Internet connectivity for Google API communication
    \item \textbf{Web Browser:} For OAuth authentication flow (Chrome, Firefox, Edge, Safari)
\end{itemize}

\subsection{Google Cloud Platform Requirements}
\begin{itemize}
    \item \textbf{Google Cloud Project:} Active GCP project with Google Sheets API enabled
    \item \textbf{OAuth 2.0 Credentials:} Client ID and Client Secret for installed application
    \item \textbf{Redirect URI:} Configured to \texttt{http://localhost:8080}
\end{itemize}

\section{Hardware Requirements}

The hardware requirements are minimal due to the efficient nature of Rust and the relatively small image size being processed.

\begin{table}[h]
\centering
\caption{Minimum and Recommended Hardware Specifications}
\label{tab:hardware}
\begin{tabular}{|l|l|l|}
\hline
\textbf{Component} & \textbf{Minimum} & \textbf{Recommended} \\
\hline
Processor & Dual-core 2.0 GHz & Quad-core 2.5 GHz or higher \\
\hline
RAM & 4 GB & 8 GB or higher \\
\hline
Storage & 500 MB free space & 1 GB free space \\
\hline
Display & 1024×768 resolution & 1920×1080 or higher \\
\hline
Network & Broadband internet & High-speed broadband \\
\hline
\end{tabular}
\end{table}

\subsection{Justification for Hardware Requirements}

\begin{itemize}
    \item \textbf{Processor:} The convolution operation requires moderate computational power. A dual-core processor can handle 400×300 pixel processing, but a quad-core provides better performance for compilation and concurrent operations.
    
    \item \textbf{RAM:} 4 GB is sufficient for basic operation, but 8 GB ensures smooth execution alongside other applications. The image data requires approximately 360 KB (400×300×3 bytes), well within these limits.
    
    \item \textbf{Storage:} The compiled binary is typically under 50 MB, with additional space needed for Cargo cache and sample images.
    
    \item \textbf{Network:} Stable internet connection is essential for OAuth flow and uploading 120,000 cells to Google Sheets. Higher bandwidth reduces upload time.
\end{itemize}

\section{System Constraints}

Several constraints influence the system design and implementation:

\subsection{Technical Constraints}
\begin{itemize}
    \item \textbf{Google Sheets Limitations:} Maximum 10 million cells per spreadsheet, though 120,000 cells are well within this limit
    \item \textbf{API Rate Limits:} Google Sheets API has quotas (100 requests per 100 seconds per user)
    \item \textbf{Image Size:} Fixed output resolution of 400×300 pixels to balance detail and rendering time
\end{itemize}

\subsection{Design Constraints}
\begin{itemize}
    \item \textbf{CLI-only Interface:} No graphical user interface, limiting accessibility for non-technical users
    \item \textbf{Single Image Processing:} Processes one image at a time rather than batch processing
    \item \textbf{Fixed Kernel:} Currently supports only one convolution kernel (sharpening)
\end{itemize}

\subsection{Operational Constraints}
\begin{itemize}
    \item \textbf{Internet Dependency:} Requires active internet connection for API access
    \item \textbf{Google Account:} Users must have a Google account for authentication
    \item \textbf{API Credentials:} Requires pre-configured Google Cloud project and OAuth credentials
\end{itemize}

This system analysis provides the foundation for understanding the technical requirements and constraints that guide the implementation phase, which is detailed in the following chapter.
